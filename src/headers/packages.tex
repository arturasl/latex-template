%%%%%%%%%%%%%%%%%%%%%%%%%%%%%%%%% useful packages

% simple string manipulations (replace in string, check if string contains substring, etc.)
\usepackage{xstring}

% adds links on citations, references, etc. (notable macros: \href{url}{text} and \url{url})
\usepackage{hyperref}

% allows to check if xetex is used to compile current file
\usepackage{ifxetex}

% gives \xspace macro which adds space unless next character is punctuation character
% useful in creating new commands with no parameters - \newcommand{\hello}{hello} would
% require to write \hello{} (or else next space would be eaten)
\usepackage{xspace}

% allows to easily define new colors (and change colors of certain document elements like
% color even rows of table (needs table option))
\usepackage[table]{xcolor}

% ads macros for text underlining (\sout{strike through}, \uline{underline}, etc.)
\usepackage{ulem}

% allows to change page color, text color, etc.; rotate, stretch text, etc.; including other
% graphical elements to generated page (png, pdf figures)
\usepackage{graphicx}

% allows to change various spacing parameters for list (itemize, enumerate)
\usepackage{tweaklist}

% allows changing way how captions are rendered (add dot after reference name, reference name,
% etc.) in various environments (tabular, float, etc.)
% also give \captionof macro which allows adding captions outside of float
\usepackage{caption}

% allows to change footnote style (reset numbering after each page, use symbols instead
% of numbers, etc.)
% stable - allow footnotes in \section{} macro
% multiple - separate multiple footnotes (\footnote{a}\footnote{b}) with comma
\usepackage[stable,multiple]{footmisc}

% lots of additional symbols (like \mathbb{N} for natural number ring)
\usepackage{amssymb}
\renewcommand{\emptyset}{\varnothing}

% writing documents with several columns (use \columnbreak for manual break).
\usepackage{multicol}

% detecting odd/even pages
% changing margins of page or text blocks
\usepackage{changepage}

% undocumented
\usepackage[retainorgcmds]{IEEEtrantools}
\usepackage{amsmath}
\usepackage{amsthm}
\usepackage{tikz}
