%%%%%%%%%%%%%%%%%%%%%%%%%%%%%%%%% new commands

% includes pdf file from obj directory (you can use ':' as
% directory separator if object is in different directory)
% 1 optional parameter - width of included pdf file
% 2 parameter - path to object relative to obj directory
\newcommand{\includeobj}[2][0.5\linewidth]{
	\ifINCLUDEOBJS
		\IfBeginWith{#2}{png:}%
			{\StrSubstitute{#2}{:}{/}[\includeobjfilename]}%
			{\StrSubstitute{obj/#2}{:}{/}[\includeobjfilename]}
		\includegraphics[width=#1,keepaspectratio]{\includeobjfilename}
	\else
		here will be ,,#1``
	\fi
}

% includes pdf file from obj directory as figure
% 1 optional parameter - width of included pdf file
% 2 parameter - path to object (see \includeobj). This path
% will be also used as label
% 3 parameter - caption (if you do not want to insert caption
% pass empty string
% example: \includefloatingobj[0.7\linewidth]{pdfimg:my_pdf}{This is my pdf}
\newcommand{\includefloatingobj}[3][0.5\linewidth]{
	\begin{figure}[h!tbp]
		\centering{
			\frame{\includeobj[#1]{#2}}%
			\ifx\\#3\\\else\caption{{#3}}\fi%
			\label{#2}%
		}
	\end{figure}
}

% example: 
% \includetwofloatingobjs%
% {gpline:pic_1}{title 1}%
% {gpline:pic_2}{title 2}%
% {full title}

\newcommand{\includetwofloatingobjs}[6][0.45]{
	\begin{figure}[h!tbp]
		\centering{
			\begin{minipage}{\linewidth}
				\begin{adjustwidth}{0em}{0em} % overwrite page margins
					\centering{
						\begin{minipage}{#1\linewidth}
							\frame{\includeobj[\linewidth]{#2}}
							\ifx\\#3\\\else\caption{{#3}}\fi
							\label{#2}
						\end{minipage}
						\qquad
						\begin{minipage}{#1\linewidth}
							\frame{\includeobj[\linewidth]{#4}}
							\ifx\\#5\\\else\caption{{#5}}\fi
							\label{#4}
						\end{minipage}
						\ifx\\#6\\\else\caption{{#6}}\fi
					}
				\end{adjustwidth}
			\end{minipage}
		}
	\end{figure}
}


\newcommand*\circled[1]{\tikz[baseline=(char.base)]{\node[shape=circle,draw,inner sep=1pt] (char) {#1};}}
\newcommand\silentsection[1]{\section*{#1}\addcontentsline{toc}{section}{#1}}

% creates a new glossary entry
% 1 parameter - name of glossary which will be used to distinguish
% this entry from other ones. Also a new macro with the same name
% will be created which will enable easier glossary linking
% 2 parameter - name
% 3 parameter - description
\newcommand{\glsDef}[3]{
  \newglossaryentry{#1} {
    name={#2},
    description={#3}
  }
  \expandafter\newcommand\csname kwd#1\endcsname[1][#1]{\emph{\glslink{#1}{##1}}\xspace}
}

% creates a new glossary and acronym entry entry
% 1 parameter - acronym (also will be used in creating macro)
% 2 parameter - name
% 3 parameter - description
\newcommand{\glsDefAndAcc}[3]{
  \newglossaryentry{def#1} {
    name={#2},
    description={#3}
  }
  \newglossaryentry{#1}{
    type=\acronymtype,
	name={#1},
    description=\glshyperlink{def#1}{#2}
  }
  \expandafter\newcommand\csname kwd#1\endcsname[1][#1]{\emph{\glslink{#1}{##1}}\glsadd{def#1}\xspace}
}

\newcommand{\citep}[2]{\cite[p. {#1}]{{#2}}} % yes, there are a lot of similar commands...
